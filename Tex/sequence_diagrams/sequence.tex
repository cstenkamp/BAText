% Demonstration of pgf-umlsd.sty, a set of convenient macros for drawing
% UML sequence diagrams. Written by Xu Yuan <xuyuan.cn AT gmail.com> from
% Southeast University, China.
% The project is hosted at Google code: http://code.google.com/p/pgf-umlsd/ 
\documentclass{article}

\usepackage{tikz}
\usepackage{pgf-umlsd}
\usepackage{geometry}
\geometry{left=0mm, right=0mm, top=0mm, bottom=0mm}
\usepgflibrary{arrows} % for pgf-umlsd
\begin{document}
	
		
\begin{figure}
	\centering
	
	
	\begin{sequencediagram}[.9]
		\tikzstyle{inststyle}+=[top color=gray!10, bottom color=blue!50, rounded corners=4mm]
		\newthread{unity}{:Unity.Client}{0}
		
		\tikzstyle{inststyle}+=[top color=gray!80, bottom color=yellow!255, rounded corners=0mm] 
		
		\newthread[blue!60][.5]{recthread}{:recThread}{10}
		\newinst{recsoc}{clt:MySocket}{0}
		\newinst{agent}{myAgent:Agent}{0}
		\newinst{inputval}{inp:InputValCont}{0}
		\newthread[red]{reclist}{:RecListThread}{3}
		\newinst{recwaitsoc}{recportsock:MySocket}{0}
		\newthread[white]{main}{main:Thread}{0}
		
		\begin{messcall}{main}{init()}{agent}{}		
		\end{messcall}		
		\begin{messcall}{main}{init()}{inputval}
		\end{messcall}
		\begin{messcall}{main}{init()}{recwaitsoc}{}		
		\end{messcall}		
		\mess{main}{start}{reclist}
		\begin{sdblock}{loop}{client==None}
			\begin{call}{reclist}{accept()}{recwaitsoc}{client}
			\end{call}
		\end{sdblock}
		\begin{call}{reclist}{accept()}{recwaitsoc}{client}
			\mess[1]{unity}{connect}{recwaitsoc}
		\end{call}		
		\begin{messcall}{reclist}{init(client)}{recsoc}{}
		\end{messcall}			
		\mess{reclist}{start}{recthread}
		\begin{sdblock}[green!20]{Run Loop}{}
			\begin{sdblock}{loop}{data==""}
				\begin{call}{recthread}{myreceive()}{recsoc}{data}
				\end{call}		
			\end{sdblock}	
			\begin{call}{recthread}{myreceive()}{recsoc}{data}
				\mess[1]{unity}{send}{recsoc}
			\end{call}					
			\begin{call}{recthread}{\small h\_s\_c(data)}{recthread}{}
				\begin{sdblock}{opt}{data}
					\begin{call}{recthread}{read()}{inputval}{result}
					\end{call}
					\begin{messcall}{recthread}{\small endEpisode(result)}{agent}
					\end{messcall}
				\end{sdblock}
			\end{call}
			\begin{messcall}{recthread}{update(data)}{inputval}{}
				\begin{call}{inputval}{\scriptsize append(data)}{inputval}{}
				\end{call}				
			\end{messcall}
			\begin{call}{recthread}{read()}{inputval}{content}
			\end{call}
			\begin{messcall}{recthread}{performAction(content)}{agent}{}
			\end{messcall}
		\end{sdblock}

		
		
		
%		\newthread[red]{sendlist}{:SendListThread}
%		\newinst{sendsoc}{sendportsock:MySocket}

	\end{sequencediagram}

\end{figure}	
	
\newpage
\clearpage

\begin{figure}
	\centering
	
	\begin{sequencediagram}
		\tikzstyle{inststyle}+=[bottom color=yellow] % custom the style
		\newthread[blue]{ss}{:SimulationServer}{0}
		\newinst{ps}{:PhysicsServer}{0}
		\newinst[2]{sense}{:SenseServer}{0}
		\newthread[red]{ctr}{:SimControlNode}{0}
		
		\begin{sdblock}[green!20]{Run Loop}{The main loop}
			\mess{ctr}{StartCycle}{ss}
			\begin{call}{ss}{Update()}{ps}{}
				\prelevel
				\begin{call}{ctr}{SenseAgent()}{ctr}{}
					\begin{call}[3]{ctr}{Read}{sense}{}
					\end{call}
				\end{call}
				\prelevel\prelevel\prelevel\prelevel
				\setthreadbias{west}
				\begin{call}{ps}{PrePhysicsUpdate()}{sense}{}
				\end{call}
				\setthreadbias{center}
				\begin{call}{ps}{Update()}{ps}{}
					\begin{call}{ps}{\small CollisionDetection()}{ps}{}
					\end{call}
					\begin{call}{ps}{Dynamics()}{ps}{}
					\end{call}
				\end{call}
				\begin{call}{ps}{PostPhysicsUpdate()}{sense}{}
				\end{call}
			\end{call}
			\mess{ss}{EndCycle}{ctr}
			\begin{call}{ctr}{ActAgent()}{ctr}{}
				\begin{call}{ctr}{Write}{sense}{}
				\end{call}
			\end{call}
		\end{sdblock}
	
	\end{sequencediagram}	
	
		
	\caption{Example of a sequence with parallel activities.}
\end{figure}


	
\end{document}