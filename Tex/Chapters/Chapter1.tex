% Chapter Template

\chapter{Introduction} % Main chapter title

\label{ChapterX} % Change X to a consecutive number; for referencing this chapter elsewhere, use \ref{ChapterX}

%----------------------------------------------------------------------------------------

% Define some commands to keep the formatting separated from the content 
\newcommand{\keyword}[1]{\textit{#1}}
\newcommand{\tabhead}[1]{\textbf{#1}}
\newcommand{\code}[1]{\texttt{#1}}
\newcommand{\file}[1]{\texttt{\bfseries#1}}
\newcommand{\option}[1]{\texttt{\itshape#1}}
\newcommand{\batchnorm}{batch normalization }
\newcommand{\Batchnorm}{Batch normalization }

%----------------------------------------------------------------------------------------
%	SECTION 1
%----------------------------------------------------------------------------------------

"sollte etwa 10\% der Gesamtarbeit ausmachen"

\section{Motivation}

Google's self-driving car, Tesla autopilot, die vision von Ubers autonomous taxis, ...

\subsection{Problem Domain}

\subsection{Goal of this thesis}

\section{Research Questions}

\section{Reading Guidelines}


\section{noclue}

As put forward by \keyword{Lex Fridman} in his MIT lecture "\keyword{Deep Learning for Self-Driving Cars}"\footnote{MIT 6.S094, course website: \url{http://selfdrivingcars.mit.edu/}} the tasks for self-driving cars can be sub-divided into the following categories: 
\begin{itemize} \bfseries
	\item Localization and Mapping
	\item Scene Understanding
	\item Movement Planning
	\item Driver State
\end{itemize}
[rrt* macht halt sowas von 3., end-to-end macht die kiste insgesamt nen bisschen anders]

Semi-autonomous vehicle components: Radar, Visible-light-camera, LIDAR, infrared-camera, stereo vision, GPS/IMU, CAN, Audio

Localization and Mapping: eg. $file:///C:/Users/Marie/Downloads/VISAPP_2015_145.pdf$
Scene Understanding/Object Detection: Scene Segentation Network (SegNet)
Movement Planning: Previously by stuff like RRT* (optimization-based control), however reinforcement learning!

End-to-End: NVIDIA Paper

dass Reinforcement Learning ja eigentlich nicht so der beste approach ist


Normally when dealing with self-driving cars, there are countless additional issues, each making up a whole new challenge on their own, like wheather conditions (snow, rain, fog), pedestrians, other cars, reflections, merging into ongoing traffic, ...
we make it easier here.

Or, according to the Torcs-paper:
The racing problem could be split into a number of different components, including
robust control of the vehicle, dynamic and static trajectory planning,
car setup, inference and vision, tactical decisions (such as overtaking) and fi-
nally overall racing strategy. With only a single car on the track, the overall
problem can be formalised as a partially observable Markov decision processes.



--> Because we only consider the case of a  single car on the track it definitely is a POMDP! seeee next chapter yay
