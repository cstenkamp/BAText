\chapter{Discussion and future directions}

In retrospect, all of the objectives of this thesis were met. The game was sucessfully transformed into an efficiently working platform for research on reinforcement learning, and several agents were implemented that learn reasonable policies. The resulting framework also makes it easy to add new agents without deep understanding of the specific implementation.

\section{Platform}

It was a long way until the setup of environment, client and server was at the state it is in the final version. The socket-connection alone has gone through many versions to come to the final version it is at right now. It is surprising how efficient this version is -- an average response time of $20-30ms$ including the TensorFlow-inference is on par with for example the delay in \keyword{openAI Universe}\footnote{\url{https://github.com/openai/universe}}, and it allows for a connection of up to $40$ FPS, which should be enough time for any agents to learn (the DQN ran for example with only $20$ FPS). In future work it would be possible to additionally increase Unity's internal \codeobj{time.TimeScale} to up to twice its current value -- for that, however, the millisecond-intervals have to be adjusted as well.

In a prior version, learning and inference were performed simultaneously in different threads. This, however, decreased the runtime by a factor of ten if both must access the same TensorFlow model. This could change if the usage of TensorFlow \codeobj{feed\_dict} was  replaced by thread-safe \codeobj{queue}s instead\footnote{see \url{https://github.com/tensorflow/tensorflow/issues/2919}}. A multi-agent solution is imaginable, in which a threaded TensorFlow graph is feeded via different threads. While it is an open question if the non-asynchronous models as presented in this thesis benefit from multithreaded feeding, it was not subsequently performed due to the increased runtime. 

In implementations where target- and online-network are updated non-continuously (like the original DQN) it is especially easy to outsource the learning thread to a remote machine, where information exchange is limited to updates of the replay memory and occasional updates of the computation graph. A further possibility to speed up the learning process could be to allow multiple Unity-clients that all aggregate samples for the same agent, stemming from the same policy. It is an open question how much delay has to be expected if server and environment run on different machines. As a major bottleneck of this implementation is samples generation, it is expected that this would increse the speed of learning drastically.

An important distinction of this platform from others is the fact that it is live, and that a user can interfere in anything an agent does. This is especially convenient to see if an agent's Q-values are reasonable. Further, a human's performance can be evaluated just like the performance of agents, also with respect to reward and Q-values as an agent would ascribe them. Having a live visual feedback of the reward (\codeother{if config.showColorArea}) has also shown to be useful functionality to assess reward quality.

A major focus point of this work was to make it as easy as possible to add further agents, models and \keyword{vectors}. A user implementing a new agent or model only needs to create a respective class that implements the interface stated in figure~\ref{fig:agentMINI} or figure~\ref{fig:modelsInt} and put it in the respective folder. Training for any new agent can be started by executing \codeother{python server.py --agent NEW\_AGENT}. All details of communicating with the environment or its memory are abstracted away by abstract classes. Adding further vectors is also easily done, as it just needs to be implemented in Unity and added to the namedtuple \codeobj{OtherInputs}. Sending all possible vectors to an agent, such that the agent decides what to make of it allows for a various amount of agents, and adding new components that rely on the environment's state is far easier than in other known environments like TORCS.

Future research will show if the agents' used observations can be improved further, for example by adding different vectors. Additionally, the agents' capacity to generalize should be tested, which could be done by adding race tracks to the game.

\section{Agents}

Most of the implemented agents were able to learn successful driving policies, even though their time to complete a track is worse than that of a human tester. This may suggest that individual components or hyperparameters are not optimized, however one has to keep in mind that no agent trained for longer than $500.000$ steps, which stands in comparison to $50.000.000$ training steps in the original DQN-paper and $2.500.000$ steps in the DDPG-paper. Due to the nature of all current deep reinforcement learning algorithms, the change of hyperparameters manifest as qualitative changes in the policy only after several hours. While temporal constraints did not allow for much further testing beyond what is depicted in this thesis, there is reason to be optimistic about the capacity of the used agents. 

For many components, there are however ways to test them more efficiently. The included \filename{gym\_test.py} script is for example a very good way to test if an implemented model works. The file expects the same interface of an agent, and as many of the gym-environments are much easier than the given game, testing can be done much faster. To test a new definition of the observation-function, it is useful to first create a supervised agent that uses this observation function, as such an agent can test much faster if the dataset is even \textit{seperable} under a certain definition of observation.

Racing games are generally an exceptionally hard case for reinforcement learning because of their highly correlated states. To encounter for example the situation of \textit{being fast in a turn}, an agent first needs to drive stable enough to speed up, which is impossible purely by chance. While it would help to simulate such situations as an agent's initial state with some probability, this approach was not considered as it does not reflect real behaviour.

Testing seemed to show that this game is in fact harder than TORCS, as agents that successfully perform in TORCS show no success in this environment. The track is especially dificult, as it starts with a long straight stretch and a very sharp tun afterwards. Many agents learned in the first epsiodes that accelerating as much as possible provides maximum reward, which leads to them driving into the wall over and over, as for example the agent using the reward function from \cite{lillicrap_continuous_2015}. Analysis of this reward-function also showed that iterations of high reward do not correspond to iterations of much progress, a further argument against such a function. The car always starting at the same track position is a likely reason for why agents are prone to repeatedly crash in the first turn. While it was demanded to not use a stochastic start state distribution, it would be interesting to see if a simpler definition of reward works better if that would be done. It has to be kept in mind however that while the current definition of reward is very complex, it appears to be somewhat reasonable, as it is the only tested reward-function that helped an agent learn sucessful policies.

There are many open ends that would probably improve the performance of agents a lot, that could not be implemented yet due to temporal constraints. A very interesting approach is for example to incorporate a pseudocount-based exploration function as suggested by \cite{ostrovski_count-based_2017} or \cite{martin_count-based_2017}. While it was tried to incorporate such a density-model into the current implementation (this is why the low-dimensional critic from figure~\ref{fig:lowdcrit} has 20 final hidden units), it did not succeed yet. It is however assumed that doing so would improve the speed of learning of agents a lot, as it can be used to force an agent to only explore in unkown situations. This would reduce the amount of training episodes where a car spontaneously steers into a wall while at full speed. Optimal would be an agent that largely follows its policy for the known part of the track, exploring only in later sections. Besides that, OpenAI recently suggested that incorporating \textit{parameter noise} as a means of providing additional exploration greatly improves performance of both implemented algorithms\footnote{\url{https://blog.openai.com/better-exploration-with-parameter-noise/}}.

Another approach that speeds up learning is \textit{prioritized experience replay}, as introduced by \cite{schaul_prioritized_2015}. Next to that, there are architectures that can wrap the implemented approaches in a multi-threaded fashion such as  A3C\cite{mnih_asynchronous_2016} or GORILA\cite{nair_massively_2015}, which would probably also greatly improve the speed of learning.

The current pace of improvements in the field of deep reinforcement learning is very quick. In the midst of writing this thesis a new technique, termed \textit{Proximal Policy Optimization} \cite{schulman_proximal_2017} surpassed the performance of the used algorithms in many MuJoCo physics-tasks. A future direction would be to implement this algorithm and compare its performance to that of the two used algorithms.

Unfortunately, this thesis did not find a way to properly incorporate pretraining into a q-learning agent, neither for DQN-based agents nor for DDPG-based agents. The implemented algorithm (\ref{alg:maketrainbatch}) enabled that q-pretraining achieves reasonable testing set accuracy, but did not speed up the learning process of actual training. It is assumed that this is because the method sets zero reward for any other action. However, another version of the \codefunc{make\_trainbatch}-function that also rewards noisy actions basing on their gaussian distance (the code for which is still in \filename{ddpg\_novision\_rl\_agent.py}) did not solve this problem either. The solution to this problem remains unknown and a matter of further research.

Due to memory constraints of the provided working station, the agents that require a \keyword{visionvector} were not tested extensively. Further research may go into this direction to test how good the performance of agents that rely purely on these vision-vectors is in comparison to agents that receive low-level information. It is also interesting to figure out what the minimal observation is that an agent requires to infer a capable driving policy. Many other approaches, like the original DDPG \cite{lillicrap_continuous_2015} or \textit{NVIDIA}s approach \cite{bojarski_end_2016} use the same visual information that a human receives as input. It is worth following the same approach in this implementation.

Furthermore, there is no good way to assess an agent's performance so far. In pre-training, the accuracy is simply taken as the percentage of actions that have the same discretized action than the supervised label or as the mean gaussian distance of those. This is similar to other approaches (e.g. \cite{bojarski_end_2016}) in that it assumes that the human performance is the ground truth of optimal performance. This approach has however several disadvantages -- for example, an agent simply driving forward already has a reasonable accuracy according to this measure, or the fact that tiny differences in steering can make the difference between driving optimally and crashing into a wall. As section~\ref{sec:resultsupervised} shows, this measure of accuracy does not correlate with actual driving performance. The only useful measure found so far therefore is the agent's progress and laptime. An optimal method to measure performance that works in both use cases would be to compare every point of the policy's trajectory with that of an optimal algorithm, like for example the introduced $RRT^*$ algorithm.

Another future direction could be to introduce rewards that change throughout the training process. At start, the agent could be rewarded for making some kind of progress, whereas only lateron its speed and laptime become more relevant. So far, it did also not matter if an agent drives a \textit{valid} round in which it stays on the track-surface at all times. A more strict punishment of driving off-track or even resetting the car if it does so are imaginable to counter this behaviour. This could also be introduced only after an agent learned a useful driving policy.

This thesis has shown that existing agents from the literature (see~\ref{sec:relatedimplements}) cannot be straight-forward applied to the implemented simulated racing platform. Further testing must show if the reason of that is the deterministic start-state as used here, or if the TORCS environment is more condoning for short-sighted action generation than this implementation. If the latter is the case, it questions if the TORCS environment is a good testbed for long-sighted action planning. Another interesting task would also be to transfer the resulted agents from this implementation, including their observation and reward-definition, to the TORCS environment to see how they perform in comparison to agents from the literature. 

Normally when dealing with self-driving cars, there are countless additional issues, each making up a whole new challenge on their own. Examples for those are wheather conditions (snow, rain, fog), pedestrians, other cars, reflections, merging into ongoing traffic, and many more situations. The situation considered here is a small sample of all situations a real car must handle. A further plan for this project is to approach more of these situations. An interesting idea is for example to introduce semantic parameters to the neural network, telling the agent information about friction-properties of the road, such that one and the same agent can transfer its knowledge to new situations. This may lead to a similar approach to \keyword{InfoGAN}\cite{chen_infogan:_2016} in the realm of reinforcement learning.

\section {Conclusion}

In conclusion, the developed framework has proven to provide a reasonable reinforcement learning environment and enables quick creation and inclusion of new agents with distinct observation- and reward functions. It is therefore suited well to serve as a research platform for reinforcement learning in self-driving cars and even showed some possible weaknesses in other such platforms, such as TORCS. 

While the optimization of in this thesis proposed functions and parameters will have to be continued by future research, they already enabled learning of reasonable driving-policies. This success was achieved even though the proposed framework seemed more difficult than average.

Due to a large amount of possible improvements that did not fit under the scope of this thesis, the author is optimistic that further optimization of the agents and their performance is possible and that the developed framework will be a useful tool for further development. 

Generally, the field of reinforcement learning is making quick progress, and the possibilities of virtual training to be applied in physical self-driving cars are endless.
It will be interesting to see, in what direction this research will go and how long it will take for reinforcement learning applications to become a norm in our everyday lives.  



%\colorbox{red}{schreiben dass superviele auf torc lernen, aber wer lernt schon auf seiner eigenen platform, thats new and cool !!!!!!!!}
%
%-wie gut dieses framework mittlerweile funktioniert, und was für ein langer weg das dahin war (dass das definitiv was war das ne ganze arbeit füllt, aber zukünftige arbeiten jetzt super gut darauf gehen, weil wegen bam framework.)
%-das das framework definitiv comparable zu dem given in den beiden torcs-papern (BEIDE NOCHMAL ZITIEREN!!) ist.
%-dass ich kein anderes paper gesehen habe dass sinvolle sachen macht... das virutal to real macht 9 discrete actions, das ddpg-torcs-ding macht "sometimes useful pocicies", ...
%
%Normally when dealing with self-driving cars, there are countless additional issues, each making up a whole new challenge on their own, like wheather conditions (snow, rain, fog), pedestrians, other cars, reflections, merging into ongoing traffic, ...
%we make it easier here.
%
%"Fragestellung aus der Einleitung wird erneut aufgegriffen und die Arbeitsschritte
%werden resümiert"
%Zusammen mit der Conclusion 10\% der Gesamtlänge



