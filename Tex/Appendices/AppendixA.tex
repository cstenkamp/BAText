% Appendix A

\newgeometry{
	a4paper,
	top=20mm,
	bottom=10mm,
	inner=24mm,
	outer=9mm,} %bindingoffset=.5cm

\lstset{
	numberblanklines=false
	,basicstyle=\ttfamily%
	,breaklines=true%
	,tabsize=1%
	,showstringspaces=false%
	,numbers=left%
	,numbersep=\lstnumbersep%
	,numberstyle=\lstnumberstyle%
	,framesep=0pt%
	,xleftmargin=\lstnumberwidth%
	,framexleftmargin=\lsthorizontalpadding%
	,xrightmargin=\lsthorizontalpadding%
	,framexrightmargin=\lsthorizontalpadding%
	,backgroundcolor=\color{verylightgray}%
	,postbreak=\ding{229}\space%
	,escapeinside={*(}{*)}
	\linespread{1.0}
}

\chapter{Comparison pseudocode \& Python-code} % Main appendix title
% https://tex.stackexchange.com/questions/22988/multicolumn-listing-for-comparison-in-latex

\label{AppendixA} % For referencing this appendix elsewhere, use \ref{AppendixA}

\vspace{-0.8cm}

\section{DQN}
\label{ap:dqn}

The following section describes the structure of an actual reinforcement learning agent, using a \textbf{Dueling Deep-Q-Network} as its model (as described in \cite{wang_dueling_2015}), performing \textbf{Double Q-learning} (as described in \cite{van_hasselt_deep_2015}). The last page consists of a comparison between the pseudocode of the general program flow of a DDQN-network (taken from \cite{mnih_human-level_2015}, with changes from \cite{van_hasselt_deep_2015} and \cite{lillicrap_continuous_2015} in blue) to the left and its corresponding python-code to the right, where each line of the pseudocode corresponds exactly to the respective line of the python-code. For information on which python- and tensorflow version are used, please see chapter~\ref{ch:program}. This code is extracted from the actual implementation within the scope of this thesis, with some changes abstracting away irrelevant details.\\

\lstinputlisting[language=Python, firstline=29]{codes/dqn.txt}
\begin{landscape}
	\begin{parcolumns}[distance=0.1em,colwidths={1=33em}]{2}
		\label{ap:dqn_comparison}
		\colchunk[1]{ \lstinputlisting[language=Pseudo]{codes/pseudo_dqn.txt}}
		\colchunk[2]{ \lstinputlisting[language=Python, lastline=27]{codes/dqn.txt} }
	\end{parcolumns}
\end{landscape}

\section{DDPG}
\label{ap:ddpg}

The following section describes the structure of an actual reinforcement learning agent, using an \textbf{actor-critic architecture} as its model, basing on the \text{Deep Deterministic Policy gradient}, as described in \cite{silver_deterministic_2014} and \cite{lillicrap_continuous_2015}. The last page consists of a comparison between the pseudocode of the general program flow of a DDPG-agent (taken from \cite{lillicrap_continuous_2015}) to the left and its corresponding python-code to the right, where each line of the pseudocode corresponds exactly to the respective line of the python-code. For information on which python- and tensorflow version are used, please see chapter+\ref{ch:program}. This code is extracted from the actual implementation within the scope of this thesis, with some changes abstracting away irrelevant details. Note that this listing does not contain a definition of a class \textit{agent}, as it is the same defined as in lines 1-48 of appendix~\ref{ap:dqn}.\\

\lstinputlisting[language=Python, firstline=33]{codes/ddpg.txt}
\begin{landscape}
	\begin{parcolumns}[distance=0.1em,colwidths={1=33em}]{2}
		\label{ap:ddpg_comparison}
		\colchunk[1]{ \lstinputlisting[language=Pseudo]{codes/pseudo_ddpg.txt}}
		\colchunk[2]{ \lstinputlisting[language=Python, lastline=32]{codes/ddpg.txt} }
	\end{parcolumns}
\end{landscape}



